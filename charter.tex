\documentclass[
11pt, % The default document font size, options: 10pt, 11pt, 12pt
codirector, % Uncomment to add a codirector to the title page
]{charter} 



% El títulos de la memoria, se usa en la carátula y se puede usar el cualquier lugar del documento con el comando \ttitle
\titulo{Análisis de espectros de relajación
mediante el uso de redes neuronales profundas}

% Nombre del posgrado, se usa en la carátula y se puede usar el cualquier lugar del documento con el comando \degreename
%\posgrado{Carrera de Especialización en Sistemas Embebidos} 
%\posgrado{Carrera de Especialización en Internet de las Cosas} 
\posgrado{Carrera de Especialización en Inteligencia Artificial}
%\posgrado{Maestría en Sistemas Embebidos} 
%\posgrado{Maestría en Internet de las cosas}

% Tu nombre, se puede usar el cualquier lugar del documento con el comando \authorname
\autor{Ing. Bruno Rais} 

% El nombre del director y co-director, se puede usar el cualquier lugar del documento con el comando \supname y \cosupname y \pertesupname y \pertecosupname
\director{Alex D. Barria C.}
\pertenenciaDirector{FIUBA} 
% FIXME:NO IMPLEMENTADO EL CODIRECTOR ni su pertenencia
\codirector{John Doe} % para que aparezca en la portada se debe descomentar la opción codirector en el documentclass
\pertenenciaCoDirector{FIUBA}

% Nombre del cliente, quien va a aprobar los resultados del proyecto, se puede usar con el comando \clientename y \empclientename
\cliente{Dr. Martín G. González}
\empresaCliente{Depto. Física FIUBA}

% Nombre y pertenencia de los jurados, se pueden usar el cualquier lugar del documento con el comando \jurunoname, \jurdosname y \jurtresname y \perteunoname, \pertedosname y \pertetresname.
\juradoUno{Nombre y Apellido (1)}
\pertenenciaJurUno{pertenencia (1)} 
\juradoDos{Nombre y Apellido (2)}
\pertenenciaJurDos{pertenencia (2)}
\juradoTres{Nombre y Apellido (3)}
\pertenenciaJurTres{pertenencia (3)}
 
\fechaINICIO{28 de febrero de 2023}		%Fecha de inicio de la cursada de GdP \fechaInicioName
\fechaFINALPlan{18 de abril de 2023} 	%Fecha de final de cursada de GdP
\fechaFINALTrabajo{11 de diciembre de 2023}	%Fecha de defensa pública del trabajo final


\usepackage[backend=biber]{biblatex}
\bibliography{referencias}
\begin{document}

\maketitle
\thispagestyle{empty}
\pagebreak


\thispagestyle{empty}
{\setlength{\parskip}{0pt}
\tableofcontents{}
}
\pagebreak


\section*{Registros de cambios}
\label{sec:registro}


\begin{table}[ht]
\label{tab:registro}
\centering
\begin{tabularx}{\linewidth}{@{}|c|X|c|@{}}
\hline
\rowcolor[HTML]{C0C0C0} 
Revisión & \multicolumn{1}{c|}{\cellcolor[HTML]{C0C0C0}Detalles de los cambios realizados} & Fecha      \\ \hline
0      & Creación del documento                                 &\fechaInicioName \\ \hline
1      & Se completa hasta el punto 5 inclusive                 & 14 de marzo de 2023 \\ \hline
2      & Se completa hasta el punto 9 inclusive					& 21 de marzo de 2023 \\ \hline
3      & Se completa hasta el punto 12 inclusive				& 28 de marzo de 2023 \\ \hline
%		  Se puede agregar algo más \newline
%		  En distintas líneas \newline
%		  Así                                                    & dd/mm/aaaa \\ \hline
%3      & Se completa hasta el punto 11 inclusive                & dd/mm/aaaa \\ \hline
%4      & Se completa el plan	                                 & dd/mm/aaaa \\ \hline
\end{tabularx}
\end{table}

\pagebreak



\section*{Acta de constitución del proyecto}
\label{sec:acta}

\begin{flushright}
Buenos Aires, \fechaInicioName
\end{flushright}

\vspace{2cm}

Por medio de la presente se acuerda con el \authorname\hspace{1px} que su 
Trabajo Final de la \degreename\hspace{1px} se titulará ``\ttitle'', consistirá 
esencialmente en implementar un método de ajuste basado en \textit{deep learning}
para obtener los parámetros necesarios para describir un material dado un
determinado modelo fenomenológico, y tendrá un presupuesto preliminar estimado de 
590 h de trabajo, con fecha de inicio \fechaInicioName\hspace{1px} 
y fecha de presentación pública el \fechaFinalName.

Se adjunta a esta acta la planificación inicial.

\vfill

% Esta parte se construye sola con la información que hayan cargado en el preámbulo del documento y no debe modificarla
\begin{table}[ht]
\centering
\begin{tabular}{ccc}
\begin{tabular}[c]{@{}c@{}}Dr. Ing. Ariel Lutenberg \\ Director posgrado FIUBA\end{tabular} & \hspace{2cm} & \begin{tabular}[c]{@{}c@{}}\clientename \\ \empclientename \end{tabular} \vspace{2.5cm} \\ 
\multicolumn{3}{c}{\begin{tabular}[c]{@{}c@{}} \supname \\ Director del Trabajo Final\end{tabular}} \vspace{2.5cm} \\
%\begin{tabular}[c]{@{}c@{}}\jurunoname \\ Jurado del Trabajo Final\end{tabular}     &  & \begin{tabular}[c]{@{}c@{}}\jurdosname\\ Jurado del Trabajo Final\end{tabular}  \vspace{2.5cm}  \\
%\multicolumn{3}{c}{\begin{tabular}[c]{@{}c@{}} \jurtresname\\ Jurado del Trabajo Final\end{tabular}} \vspace{.5cm}                                                                     
\end{tabular}
\end{table}




\section{1. Descripción técnica-conceptual del proyecto a realizar}
\label{sec:descripcion}

%Conocer las propiedades eléctricas y mecánicas de materiales es importante para 
%la investigación, asi como para diversas aplicaciones. El desarrollo de nuevos materiales,
%el control de calidad y el mantenimiento preventivo son algunos ejemplos de la importancia
%de saber las propiedades mencionadas. 

El análisis de espectros de relajación es esencial para el conocimiento de las propiedades
eléctricas y mecánicas de materiales. Para describir los procesos de relajación 
se utilizan modelos fenomenológicos que permiten caracterizar la respuesta mecánica 
y eléctrica del material a partir de un pequeño conjunto de parámetros.
Estos parámetros son relevantes para la investigación básica de materiales, como para el
diseño de dispositivos y otras aplicaciones. Por lo tanto, es importante poder establecer la cantidad 
de procesos de relación presentes en el espectro y, dentro de los modelos fenomenológicos disponibles,
definir cuál es el más adecuado a utilizar para la descripción del espectro.

Sin embargo, la obtención de algunas características en los materiales es un desafío para 
aplicaciones de alta temperatura y alta energía. La aparición de la temática de \textit{deep learning}
ha despertado el interés en la comunidad científica para la aplicación en multiples campos de la ciencia y tecnología. 
En particular, los esquemas guiados por datos (\textit{data-driven}), son alternativas populares y
poderosas para construir modelos para la predicción de propiedades y el diseño de
materiales, acelerando en gran medida el descubrimiento y la aplicación de nuevos materiales.


Bajo este contexto, el grupo de Láser, Óptica de Materiales y Aplicaciones 
Electromagnéticas de la Facultad de Ingeniería de la Universidad de Buenos Aires 
está interesado en implementar un método de ajuste basado en \textit{deep learning} para 
obtener los parámetros necesarios para describir un material, dado un determinado 
modelo fenomenológico, comparando el rendimiento contra un método tradicional y 
otro bayesiano de obtención de parámetros. Por esta razón, se propuso como trabajo 
final para la Carrera de Especialización en Inteligencia Artificial a través del 
programa de vinculación con empresas.

En la figura 1 se muera el diagrama en bloques del sistema. La entrada consistirá 
en modelos fenomenológicos con sus respectivas etiquetas. Luego se entrenará un 
algoritmo de \textit{deep learning} y con ello se predecirán las características de los
materiales para compararlas con métodos tradicionales de obtención de estas 
características.
\begin{figure}[htpb]
	\centering 
	\includegraphics[width=1\textwidth]{./Figuras/Workflow.png}
	\caption{Diagrama en bloques del sistema.}
	\label{fig:diagBloques}
\end{figure}

\vspace{25px}

\section{2. Identificación y análisis de los interesados}
\label{sec:interesados}

 

\begin{table}[ht]
%\caption{Identificación de los interesados}
%\label{tab:interesados}
\begin{tabularx}{\linewidth}{@{}|l|X|X|l|@{}}
\hline
\rowcolor[HTML]{C0C0C0} 
Rol           & Nombre y Apellido & Organización 	& Puesto 	\\ \hline
Cliente       & \clientename      &\empclientename	&     -   	\\ \hline
Responsable   & \authorname       & FIUBA        	& Alumno 	\\ \hline
Colaboradores & Dra. Ligia Ciocci Brazzano      & Grupo de Láser, Óptica de Materiales y Aplicaciones Electromagnéticas de la FIUBA &      -  	\\ \hline
Orientador    & \supname	      & \pertesupname 	& Director Trabajo final \\ \hline
Usuario final & Comunidad científica                  & Grupo de Láser, Óptica de Materiales y Aplicaciones Electromagnéticas de la FIUBA &    -    	\\ \hline
\end{tabularx}
\end{table}

Características de los interesados:

\begin{itemize}
	\item Cliente: aportara conocimiento y análisis de los resultados.
	\item Colaborador: brindará soporte en temas relacionados con la física del proyecto 
	y análisis de los resultados.
	\item Orientador: se le informará sobre los avances del proyecto. Brindará conocimiento
	y ayuda en el aspecto técnico de IA.    
\end{itemize}



\section{3. Propósito del proyecto}
\label{sec:proposito}

El propósito de este proyecto es colaborar con el grupo de investigación de la Facultad 
de Ingeniería de la Universidad de Buenos Aires especializado en Láser, Óptica de Materiales 
y Aplicaciones Electromagnéticas aportando una nueva alternativa de construcción de modelos 
para la predicción de propiedades eléctricas y mecánicas de materiales.

\section{4. Alcance del proyecto}
\label{sec:alcance}

El presente proyecto comprende:

\begin{itemize}
	\item Diseño de un método de \textit{deep learning} que permita predecir las propiedades de materiales.
	\item Estudio del método aplicándolo a espectros simulados y medidos con ruido y relajaciones superpuestas.
	\item Comparación de los resultados con el método tradicional y el bayesiano.
	\item El usuario final tendrá acceso al código libremente.
\end{itemize}


Por otro lado no queda comprendido:

\begin{itemize}
	\item La ejecución y el mantenimiento del software entregado luego de la finalización del trabajo.
\end{itemize}





\section{5. Supuestos del proyecto}
\label{sec:supuestos}


Para el desarrollo del presente proyecto se supone que:

\begin{itemize}
	\item Se podrá acceder a la información y/o bases de datos.
	\item Las 600 h serán suficientes para cumplir con el objetivo del proyecto.
	\item Se aplicarán conocimientos adquiridos en la Carrera de Especialización en Inteligencia Artificial.
	\item Se brindará la bibliografía necesaria para entender y aplicar los fenómenos físicos alcanzables a este proyecto.
	\item Las complicaciones que puedan surgir no atrasarán a la entrega estipulada del proyecto.
\end{itemize}


\section{6. Requerimientos}

\label{sec:requerimientos}

\begin{enumerate}
	\item Requerimientos funcionales.
		\begin{enumerate}
			\item El modelo final propuesto debe ser capaz de predecir la características
			de un material definidas por el cliente.
			\item El sistema debe poder soportar el volumen de datos entregados para el
			entrenamiento.
			\item Se deben probar al menos 3 algoritmos distintos de \textit{deep learning}.
			\item El lenguaje de desarrollo debe ser en Python.
		\end{enumerate}
	\item Requerimientos testing.
		\begin{enumerate}
			\item Los resultados serán provistos al cliente y el evaluará la eficacia de los
			modelos desarrollados.
			\item  Para el testeo se debe contar con un ambiente estable y con los 
			\textit{datasets} provistos por el cliente. 
		\end{enumerate}
	\item Requerimientos de documentación.
		\begin{enumerate}
			\item El código documentado y entregado debe ser claro, replicable y sustentable.
			\item Un manual que contenga cómo se realiza
			la implementación y la configuración del código.
		\end{enumerate}
\end{enumerate}




\section{7. Historias de usuarios (\textit{Product backlog})}
\label{sec:backlog}




Para el cálculo de los \textit{story points} correspondientes a los
usuarios se tendrán en cuenta los siguientes criterios:

\begin{itemize}
	\item Dificultad: evalúa la cantidad de trabajo a realizar.
	\item Complejidad: hace referencia al nivel de sofisticación del trabajo.
	\item Incertidumbre: se refiere al nivel de riego que involucra realizar
	la tarea. 
\end{itemize}

El peso que se le asignará a cada criterio quedan detallado en la 
siguiente tabla.


\begin{table}[H]
	%\caption{Identificación de los interesados}
	%\label{tab:interesados}
	\begin{tabularx}{\linewidth}{@{}|l|X|X|l|@{}}
	\hline
	\rowcolor[HTML]{C0C0C0} 
	Peso   & Dificultad & Complejidad 	& Incertidumbre 	\\ \hline
	Bajo   & 1      & 2	&  2   	\\ \hline
	Medio  & 3      & 5 &  3	\\ \hline
	Alto   & 5      & 8 &  5  	\\ \hline
	\end{tabularx}
\end{table}


Las historias identificadas son las siguientes:

\begin{itemize}
	\item Historia 1: como desarrollador quiero estudiar el aspecto físico
	de lo que se requiere en el proyecto. Dificultad: 1; complejidad: 5; incertidumbre: 3.  \textbf{\textit{Story point}: 13}.
	\item Historia 2: como desarrollador quiero aprender los modelos de \textit{deep learning} 
	aplicables al proyecto. Dificultad: 3; complejidad: 5; incertidumbre: 3.  \textbf{\textit{Story point}: 13}.
	\item Historia 3: como desarrollador quiero aplicar los modelos aprendidos de \textit{deep learning} 
	a la predicción de características de materiales.
	Dificultad: 3; complejidad: 8; incertidumbre: 5. \textbf{\textit{Story point}: 21}.
	\item Historia 4: como cliente quiero evaluar la predicción de los modelos respecto
	a las mediciones reales. 
	Dificultad: 1; complejidad: 2; incertidumbre: 2. \textbf{\textit{Story point}: 5}.
	\item Historia 5: como usuario final quiero utilizar una herramienta de IA para
	conocer las características de materiales. 
	Dificultad: 1; complejidad: 2; incertidumbre: 5. \textbf{\textit{Story point}: 8}.
\end{itemize}


\section{8. Entregables principales del proyecto}

Los entregables del proyecto son:

\begin{enumerate}
	\item Plan de trabajo del proyecto.
	\item Informe de avance.
	\item Documentación de los modelos/herramientas de inteligencia artificial utilizadas.
	\item Código fuente.
	\item Resultados de las pruebas llevadas a cabo y comparación con las características reales de los materiales.
	\item Manual de implementación del código.
	\item Memoria del Trabajo Final.
\end{enumerate}


\section{9. Desglose del trabajo en tareas}
\label{sec:wbs}



\begin{enumerate}
\item Capacitación general. (55 h)
	\begin{enumerate}
	\item Búsqueda y estudio de material relacionado con las características
	de los materiales. (15 h)
	\item Búsqueda y estudio relacionado con modelos de \textit{deep learning}.
	(40 hs)
	\end{enumerate}
\item Planificación. (55 h)
	\begin{enumerate}
	\item Realizar el plan de proyecto. (25 h)
	\item Definir del alcance, requerimientos y tareas (15 h)
	\item Estimar tiempos, recursos humanos y presupuesto. (15 h)
	\end{enumerate}
\item Implementación y diseño de modelos. (215 h)
	\begin{enumerate}
	\item Preparar el ambiente de diseño y prueba de códigos. (10 h)
	\item Obtener \textit{dataset}. (5 h)
	\item Preparar datos para el entrenamiento. (40 h)
	\item Definir modelos a utilizar. (30 h)
	\item Entrenar modelos estudiados. (35 h)
	\item Realizar ajustes en los modelos. (15 h)
	\item Testear el código unitariamente. (30 h)
	\item Revisar código integralmente. (10 h)
	\item Optimizar código. (40 h)
	\end{enumerate}
\item Análisis de resultados. (100 h)
	\begin{enumerate}
	\item Analizar resultados individuales. (30 h)
	\item Documentar resultados. (40 h)
	\item Discutir resultados con el cliente. (30 h)
	\end{enumerate}
\item Documentación del producto. (35 h)
	\begin{enumerate}
		\item Subir material al repositorio de forma prolija. (5 h)
		\item Desarrollar el manual de uso y configuración del modelo. (30 h)
	\end{enumerate}
\item Presentación del trabajo. (130 h)
	\begin{enumerate}
		\item Escribir el informe de avance. (30 h)
		\item Escritura de memoria. (40 h)
		\item Revisión y correcciones de la memoria. (20 h)
		\item Elaboración del video demo. (10 h)
		\item Preparar presentación publica. (20 h)
		\item Revisión y correcciones de la presentación (10 h)
	\end{enumerate}
\end{enumerate}

Cantidad total de horas: (590 h)



\section{10. Diagrama de Activity On Node}
\label{sec:AoN}

En la siguiente figura se muestra el diagrama \textit{Activity On Node}
del proyecto. Las flechas con mayor grosor representa el camino
crítico y la duración de las tareas están representadas en horas. Pos su
parte, los colores representan el conjunto de tareas desglosado en la sección
anterior.


\begin{figure}[htpb]
\centering 
\includegraphics[width=.9\textwidth]{./Figuras/Diagrama AonNode.png}
\caption{Diagrama de \textit{Activity on Node}.}
\label{fig:AoN}
\end{figure}


\section{11. Diagrama de Gantt}
\label{sec:gantt}

\begin{consigna}{red}

Existen muchos programas y recursos \textit{online} para hacer diagramas de Gantt, entre los cuales destacamos:

\begin{itemize}
\item Planner
\item GanttProject
\item Trello + \textit{plugins}. En el siguiente link hay un tutorial oficial: \\ \url{https://blog.trello.com/es/diagrama-de-gantt-de-un-proyecto}
\item Creately, herramienta online colaborativa. \\\url{https://creately.com/diagram/example/ieb3p3ml/LaTeX}
\item Se puede hacer en latex con el paquete \textit{pgfgantt}\\ \url{http://ctan.dcc.uchile.cl/graphics/pgf/contrib/pgfgantt/pgfgantt.pdf}
\end{itemize}

Pegar acá una captura de pantalla del diagrama de Gantt, cuidando que la letra sea suficientemente grande como para ser legible. 
Si el diagrama queda demasiado ancho, se puede pegar primero la ``tabla'' del Gantt y luego pegar la parte del diagrama de barras del diagrama de Gantt.

Configurar el software para que en la parte de la tabla muestre los códigos del EDT (WBS).\\
Configurar el software para que al lado de cada barra muestre el nombre de cada tarea.\\
Revisar que la fecha de finalización coincida con lo indicado en el Acta Constitutiva.

En la figura \ref{fig:gantt}, se muestra un ejemplo de diagrama de Gantt realizado con el paquete de \textit{pgfgantt}. En la plantilla pueden ver el código que lo genera y usarlo de base para construir el propio.

\begin{figure}[htbp]
\begin{center}
\begin{ganttchart}{1}{12}
  \gantttitle{2020}{12} \\
  \gantttitlelist{1,...,12}{1} \\
  \ganttgroup{Group 1}{1}{7} \\
  \ganttbar{Task 1}{1}{2} \\
  \ganttlinkedbar{Task 2}{3}{7} \ganttnewline
  \ganttmilestone{Milestone o hito}{7} \ganttnewline
  \ganttbar{Final Task}{8}{12}
  \ganttlink{elem2}{elem3}
  \ganttlink{elem3}{elem4}
\end{ganttchart}
\end{center}
\caption{Diagrama de Gantt de ejemplo}
\label{fig:gantt}
\end{figure}


\begin{landscape}
\begin{figure}[htpb]
\centering 
\includegraphics[height=.85\textheight]{./Figuras/Gantt-2.png}
\caption{Ejemplo de diagrama de Gantt rotado}
\label{fig:diagGantt}
\end{figure}

\end{landscape}

\end{consigna}


\section{12. Presupuesto detallado del proyecto}
\label{sec:presupuesto}


\begin{table}[htpb]
\centering
\begin{tabularx}{\linewidth}{@{}|X|c|r|r|@{}}
\hline
\rowcolor[HTML]{C0C0C0} 
\multicolumn{4}{|c|}{\cellcolor[HTML]{C0C0C0}COSTOS DIRECTOS} \\ \hline
\rowcolor[HTML]{C0C0C0} 
Descripción &
  \multicolumn{1}{c|}{\cellcolor[HTML]{C0C0C0}Cantidad} &
  \multicolumn{1}{c|}{\cellcolor[HTML]{C0C0C0}Valor unitario [ARS]} &
  \multicolumn{1}{c|}{\cellcolor[HTML]{C0C0C0}Valor total [ARS]} \\ \hline
 Horas del responsable & 
  \multicolumn{1}{c|}{590} &
  \multicolumn{1}{c|}{3.000} &
  \multicolumn{1}{c|}{1.770.000} \\ \hline
Horas de los colaboradores &
  \multicolumn{1}{c|}{150} &
  \multicolumn{1}{c|}{3.000} &
  \multicolumn{1}{c|}{450.000} \\ \hline
\multicolumn{3}{|c|}{SUBTOTAL} &
  \multicolumn{1}{c|}{2.220.000} \\ \hline
\rowcolor[HTML]{C0C0C0} 
\multicolumn{4}{|c|}{\cellcolor[HTML]{C0C0C0}COSTOS INDIRECTOS} \\ \hline
\rowcolor[HTML]{C0C0C0} 
Descripción &
  \multicolumn{1}{c|}{\cellcolor[HTML]{C0C0C0}Cantidad} &
  \multicolumn{1}{c|}{\cellcolor[HTML]{C0C0C0}Valor unitario [ARS]} &
  \multicolumn{1}{c|}{\cellcolor[HTML]{C0C0C0}Valor total [ARS]} \\ \hline
\multicolumn{1}{|l|}{Capacitación del responsable} &
   1 &
   \multicolumn{1}{c|}{403.900} &
   \multicolumn{1}{c|}{403.900} \\ \hline
\multicolumn{1}{|l|}{Equipo informático} &
   1 &
   \multicolumn{1}{c|}{200.000} &
   \multicolumn{1}{c|}{200.000} \\ \hline
\multicolumn{1}{|l|}{Internet} &
   10 &
   \multicolumn{1}{c|}{4.500}&
   \multicolumn{1}{c|}{45.000}\\ \hline
\multicolumn{3}{|c|}{SUBTOTAL} &
  \multicolumn{1}{c|}{648.900} \\ \hline
\rowcolor[HTML]{C0C0C0}
\multicolumn{3}{|c|}{TOTAL} &
\multicolumn{1}{c|}{2.868.900}\\ \hline
\end{tabularx}%
\end{table}


\section{13. Gestión de riesgos}
\label{sec:riesgos}

\begin{consigna}{red}
a) Identificación de los riesgos (al menos cinco) y estimación de sus consecuencias:
 
Riesgo 1: detallar el riesgo (riesgo es algo que si ocurre altera los planes previstos de forma negativa)
\begin{itemize}
	\item Severidad (S): mientras más severo, más alto es el número (usar números del 1 al 10).\\
	Justificar el motivo por el cual se asigna determinado número de severidad (S).
	\item Probabilidad de ocurrencia (O): mientras más probable, más alto es el número (usar del 1 al 10).\\
	Justificar el motivo por el cual se asigna determinado número de (O). 
\end{itemize}   

Riesgo 2:
\begin{itemize}
	\item Severidad (S): 
	\item Ocurrencia (O):
\end{itemize}

Riesgo 3:
\begin{itemize}
	\item Severidad (S): 
	\item Ocurrencia (O):
\end{itemize}


b) Tabla de gestión de riesgos:      (El RPN se calcula como RPN=SxO)

\begin{table}[htpb]
\centering
\begin{tabularx}{\linewidth}{@{}|X|c|c|c|c|c|c|@{}}
\hline
\rowcolor[HTML]{C0C0C0} 
Riesgo & S & O & RPN & S* & O* & RPN* \\ \hline
       &   &   &     &    &    &      \\ \hline
       &   &   &     &    &    &      \\ \hline
       &   &   &     &    &    &      \\ \hline
       &   &   &     &    &    &      \\ \hline
       &   &   &     &    &    &      \\ \hline
\end{tabularx}%
\end{table}

Criterio adoptado: 
Se tomarán medidas de mitigación en los riesgos cuyos números de RPN sean mayores a...

Nota: los valores marcados con (*) en la tabla corresponden luego de haber aplicado la mitigación.

c) Plan de mitigación de los riesgos que originalmente excedían el RPN máximo establecido:
 
Riesgo 1: plan de mitigación (si por el RPN fuera necesario elaborar un plan de mitigación).
  Nueva asignación de S y O, con su respectiva justificación:
  - Severidad (S): mientras más severo, más alto es el número (usar números del 1 al 10).
          Justificar el motivo por el cual se asigna determinado número de severidad (S).
  - Probabilidad de ocurrencia (O): mientras más probable, más alto es el número (usar del 1 al 10).
          Justificar el motivo por el cual se asigna determinado número de (O).

Riesgo 2: plan de mitigación (si por el RPN fuera necesario elaborar un plan de mitigación).
 
Riesgo 3: plan de mitigación (si por el RPN fuera necesario elaborar un plan de mitigación).

\end{consigna}


\section{14. Gestión de la calidad}
\label{sec:calidad}

\begin{consigna}{red}
Para cada uno de los requerimientos del proyecto indique:
\begin{itemize} 
\item Req \#1: copiar acá el requerimiento.

\begin{itemize}
	\item Verificación para confirmar si se cumplió con lo requerido antes de mostrar el sistema al cliente. Detallar 
	\item Validación con el cliente para confirmar que está de acuerdo en que se cumplió con lo requerido. Detallar  
\end{itemize}

\end{itemize}

Tener en cuenta que en este contexto se pueden mencionar simulaciones, cálculos, revisión de hojas de datos, consulta con expertos, mediciones, etc.  Las acciones de verificación suelen considerar al entregable como ``caja blanca'', es decir se conoce en profundidad su funcionamiento interno.  En cambio, las acciones de validación suelen considerar al entregable como ``caja negra'', es decir, que no se conocen los detalles de su funcionamiento interno.

\end{consigna}

\section{15. Procesos de cierre}    
\label{sec:cierre}

\begin{consigna}{red}
Establecer las pautas de trabajo para realizar una reunión final de evaluación del proyecto, tal que contemple las siguientes actividades:

\begin{itemize}
	\item Pautas de trabajo que se seguirán para analizar si se respetó el Plan de Proyecto original:
	 - Indicar quién se ocupará de hacer esto y cuál será el procedimiento a aplicar. 
	\item Identificación de las técnicas y procedimientos útiles e inútiles que se emplearon, y los problemas que surgieron y cómo se solucionaron:
	 - Indicar quién se ocupará de hacer esto y cuál será el procedimiento para dejar registro.
	\item Indicar quién organizará el acto de agradecimiento a todos los interesados, y en especial al equipo de trabajo y colaboradores:
	  - Indicar esto y quién financiará los gastos correspondientes.
\end{itemize}

\end{consigna}

\printbibliography
\end{document}
